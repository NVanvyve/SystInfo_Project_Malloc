\documentclass[10pt,a4paper]{article}
\usepackage[utf8]{inputenc}
\usepackage[french]{babel}
\usepackage[T1]{fontenc}
\usepackage{amsmath}
\usepackage{amsfonts}
\usepackage{amssymb}
\usepackage{makeidx}
\usepackage{graphicx}
\usepackage{lmodern}

\author{Aigret Julien\and Vanvyve Nicolas}
\date{16.03.2016}
\title{LSINF1252\\Rapport de projet 1 :\\Implémentation de malloc, calloc et free}
\begin{document}
\maketitle

%\tableofcontents %Pas obligatoire

\section{Gestion de la mémoire}
\subsection{Initialisation}
Nous avons fait le choix d'initialiser notre heap à 1MB, taille qui semble raisonnable pour notre utilisation actuelle, mais qui eut être modifiée si besoin est.

\subsection{Gestion de la fragmentation}
Dans notre programme, nous avons fait le choix d'allouer immédiatement une zone de mémoire (ou une combinaison de zones de mémoire) si elle est de la taille demandée. Dans le cas où il n'y a pas
\section{Tests comparatifs}



%Consignes

%Vous devez écrire un rapport au format PDF d’une ou deux pages, contenant les ex-
%plications de votre implémentation, les difficultés rencontrées, les cas couverts par vos
%tests unitaires ainsi qu’une figure montrant les résultats d’une mesure de performance
%de votre implémentation.

% Mesure de performance : comparer malloc et mymalloc

\end{document}
