\documentclass[10pt,a4paper]{article}
\usepackage[utf8]{inputenc}
\usepackage[french]{babel}
\usepackage[T1]{fontenc}
\usepackage{amsmath}
\usepackage{amsfonts}
\usepackage{amssymb}
\usepackage{makeidx}
\usepackage{graphicx}
\usepackage{lmodern}
\usepackage{verbatim}


\author{Aigret Julien \texttt{(8343-13-00)}\and Vanvyve Nicolas \texttt{(6590-13-00)}}
\date{16 mars 2016}
\title{LSINF1252\\Rapport de projet 1 :\\Implémentation de malloc, calloc et free}

\begin{document}

\maketitle

\section{Gestion de la mémoire}
\subsection{Initialisation}
Nous avons fait le choix d'initialiser notre heap à 1MB, taille qui semble raisonnable pour notre utilisation actuelle, mais qui eut être modifiée si besoin est. Cette limite n'est pas modifiée tout au long de l'éxécution de notre code.

\subsection{Fragmentation}
Dans notre programme, nous avons fait le choix d'allouer immédiatement une zone de mémoire (ou une combinaison de zones de mémoire) si elle est de la taille demandée par l'utilisateur. Dans le cas où il n'y a pas de zone de mémoire de taille précise, deux cas de figure se présentent : soit une ou des zones plus grandes ont été trouvées, ou aucune de taille suffisante n'a été trouvée. Dans le cas où une plus grande zone a été trouvée, nous gardons la plus grande de toutes et y installons notre zone de mémoire, créant un nouveau blockheader à la fin de celle-ci pointant sur la fin de la "zone libre".


\section{Tests unitaires}

Nous avons implémenté les tests unitaires décrits ci dessous via CUnit.
\begin{itemize}
	\item [$\bullet$] Myfree desalloc: nous vérifions que \texttt{alloc} de \texttt{block\_header} vaut bien 0.
	\item [$\bullet$] Mymalloc alloc : idem que ci dessus mais \texttt{alloc} de \texttt{block\_header} doit valoir 1
	\item [$\bullet$] Mycalloc alloc : idem que Mymalloc alloc
	\item [$\bullet$] Mymalloc size : nous vérifions que l'espace alloué correspond bien au multiple de 4 supérieur ou égal(en byte) de celui demandé.
	\item [$\bullet$] Mymalloc NULL : nous vérifions que si 0 byte sont demandé \texttt{mymalloc} renvoie \texttt{NULL}
	\item [$\bullet$] Mymalloc 2 alloc : nous vérifions que deux allocation successive via \texttt{mymalloc} ne pointe pas vers la meme adresse
	\item [$\bullet$] Mycalloc initialisation : nous vérifions que après un appel à \texttt{mycalloc} la mémoire allouée est bien initialisée à 0
\end{itemize}
Les test qui s'applique à \texttt{mymalloc} aurait pu etre transformés pour \texttt{mycalloc}. Mais étant donné que nous avons utilisé \texttt{mymalloc} pour implémenter le second cela ne servirait à rien.


\section{Tests comparatifs}

Nous constatons, comme illustré sur les graphe (fig !! Ref needed) que nos fonctions sont plus rapides pour les faibles allocations de mémoires, mais plus la mémoire devient grande et plus elles prennent du temps. Cela est certainelent dû aux algorithmes de recherches utilisés par les fonctions built-in qui doivent être optimisés pour le grandes zones de mémoire. Notre initialisation de la zone mémoire par calloc est aussi très rudimentaire, et prend de plus en plus de temps au fur et à mesure que la taille de la zone à initialiser augmente.


\end{document}
%Consignes
%Vous devez écrire un rapport au format PDF d’une ou deux pages, contenant les ex-
%plications de votre implémentation, les difficultés rencontrées, les cas couverts par vos
%tests unitaires ainsi qu’une figure montrant les résultats d’une mesure de performance
%de votre implémentation.
% Mesure de performance : comparer malloc et mymalloc
